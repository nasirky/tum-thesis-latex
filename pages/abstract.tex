\chapter{\abstractname}

Delay Tolerant Networks (DTN) allows asynchronous content sharing between different devices. The basic idea behind DTN is to communicate data/messages from the source to the destination without having complete network connectivity. The Opportunistics Network Environment (also known as ONE) simulator is a tool for evaluating DTNs. The ONE simulator provides us with a number of different movement models and routing algorithms to simulate various conditions. One of the interesting features is the ability to provide Map-based movement which enables us to model any area of the map. The ONE simulator comes with the map of Helsinki by default.
The ONE simulator lacks support for Access Points based connectivity, where a node can either be in an access point, station adapter or ad-hoc mode. There are some rules governing the connectivity between these modes such as station adapter can only connect to access point and ad-hoc nodes can only connect to other ad-hoc nodes. The main goal of this thesis is evaluating how content sharing can be improved (or will be impacted) by access point as ad-hoc networking is not real yet. For this evaluation, we need to extend the ONE simulator to support access points. A good way to simulate is by modeling the list of publicly available access points on the map of a city (which is Munich city in our case) and then evaluate the effect of access points on the availability of the messages.
The ONE simulator also supports routing of the floating content. Floating content is an ephemeral content sharing service, solely dependent on mobile devices in the vicinity using principles of opportunistic networking. One of the main characteristics of the floating content is its area of availability, also known as the availability/anchor zone. Any node going outside the availability area of the message would drop copies of that message.
Since floating content is only available inside the anchor zone, where they can be replicated to other nodes, sometimes there are either not enough nodes or there are a lot of nodes in the anchor zone and the information may die out quicker than anticipated. One way of increasing the availability zone of the message is to snap the messages to the nearest \textbf{k} anchor points. The concept is that whenever a message is copied to the access point, the center of that message is set to be the location of the access point, thus effectively changing the center and anchor zone of that copy of the message. Theoretically, this would increase the availability of the message, however, being a DTN there is no such guarantee. The second main task is to implement the snapping to access point features and then run simulations to evaluate the effect of snapping to access points on the life (duration of availability) of the message.

% !TeX root = ../main.tex
% Add the above to each chapter to make compiling the PDF easier in some editors.

\chapter{Introduction}\label{chapter:introduction}

\section{Motivation}
\newpage
\section{Research Questions/Goals}
Several questions/goals were explored in this work to weigh different extensions to the ONE Simulator such as adding access point functionality as well as snapping to access points etc. We have formulated these goals/research questions below :\newline
\newline\textbf{Q1: Modeling a city and exploring the impact of access points on the performance of geo-based information sharing.}\newline
The ONE Simulator currenly models Helsinki City and doesn't have any access points feature. The main idea here is to Model Munich City (especially the central part), Implement Access Points and then explore the impact of the access points on the geo-based information sharing performance using ONE Simulator. It also involves mapping publicly available Access Points on the Munich Model City Map.\newline
\newline\textbf{Q2: How does snapping to access point affect the availabilty of a floating message in a Delay Tolerant Network (DTN)?}\newline
\textit{Snapping to Access Point} is the concept of increasing the availability of the message beyond its original availability zone. The idea is to change the center of the message when it is encounters an access point. However, we do limit the number of access points a message can snap to. After implementation of snapping to access point, we will test how does the snapping to different number of access points affect the availability of the floating message in a DTN.
